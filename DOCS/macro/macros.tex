\newcommand{\modif}[1]{\begin{color}{red}#1\end{color}} % To highlight changes
\newcommand{\modiff}[1]{\begin{color}{blue}#1\end{color}} % To highlight changes
\newcommand{\remove}[1]{\begin{color}{green}#1\end{color}} % To highlight changes
%\newcommand{\remove}[1]{} % To highlight changes

\newcommand{\rulelabel}[1]{\textit{\small{(#1)}}} % label for operational semantics rules

%%% Environments and theorems %%%
%\newtheorem{theorem}{Theorem}[section]
%\newtheorem{lemma}{Lemma}[section]
%\newtheorem{corollary}{Corollary}[section]
%\newtheorem{property}{Property}[section]
%\newtheorem{definition}{Definition}[section]
%\newtheorem{remark}{Remark}[section]

%%% Colors %%%
\definecolor{mygray}{rgb}{.90,.90,.90}
\definecolor{myDarkGray}{rgb}{.75,.75,.75}
\definecolor{myLightGray}{rgb}{.965,.965,.965}

\newcommand{\graybg}[1]{\colorbox{mygray}{#1}} % To highlight some parts of a specification

%%% Languges names

\newcommand{\SCEL}{\textsc{SCEL}}
\newcommand{\SCELt}{\textsc{SCEL2}}
\newcommand{\Klaim}{\textsc{Klaim}}
\newcommand{\MetaKlaim}{\textsc{MetaKlaim}}
\newcommand{\kos}{$\mathcal{K}$oS}
\newcommand{\kaos}{\textsc{Kaos}}
\newcommand{\cows}{\textsc{COWS}}
\newcommand{\ccpi}{\textsc{cc-pi}}
\newcommand{\pic}{$\pi$-calculus}

\newcommand{\xmltag}[1]{\mathtt{<\!#1\!>}}
\newcommand{\und}{\_}                       % underscore
\newcommand{\dontcare}{\mathtt{\und}}
\newcommand{\arr}[1]{\langle #1 \rangle}    % for tuples of objects
\newcommand{\arrcomp}{:}
\newcommand{\set}[1]{\{#1\}}
\newcommand{\undef}{\mathbf{undef}} % denota undefined
\newcommand{\false}{\mathbf{false}}
\newcommand{\true}{\mathbf{true}}
\newcommand{\assoc}[2]{#1 \mapsto #2} % associazione variabile-valore
\newcommand{\substi}[1]{\{#1\}}

\newcommand{\delimite}[1]{\mbox{$\overstackrel{\belowfill}{\mbox{$\overstackrel{#1}{\abovefill}$}}$}}
\makeatletter
\def \abovefill{
  {\vrule width0.3mm height 1.8mm depth-0.3mm}
  \leaders\hrule height1.8mm depth-1.5mm\hfill
  {\vrule width0.3mm height 1.8mm depth-0.3mm}}
\makeatother
\makeatletter
\def \belowfill{
  {\vrule width0.3mm height1.5mm}
  \leaders\hrule height0.3mm\hfill
  {\vrule width0.3mm height1.5mm}}
\makeatother

%%% SYNTAX

\newcommand{\actell}{\textbf{tell}}                           %output operation
\newcommand{\acretract}{\textbf{retract}}
\newcommand{\actellp}[4]{\ensuremath{\mathbf{tell}(#1,#2,#3)@#4}}
\newcommand{\acretractp}[4]{\ensuremath{\mathbf{retract}(#1,#2,#3)@#4}}
\newcommand{\acput}{\textbf{put}}                           %output operation
\newcommand{\acout}{\textbf{out}}                           %output operation
\newcommand{\acputp}[2]{\ensuremath{\mathbf{put}(#1)@#2}}   %output operation with parameters
\newcommand{\acputpq}[3]{\ensuremath{\mathbf{put}(#1)@_{#3}#2}}   %output operation with parameters and quality level
\newcommand{\acget}{\textbf{get}}
\newcommand{\acin}{\textbf{in}}
\newcommand{\acgetp}[2]{\ensuremath{\mathbf{get}(#1)@#2}}
\newcommand{\acreadold}{\textbf{read}}
%\newcommand{\acreadp}[2]{\ensuremath{\mathbf{read}(#1)@#2}}
%\newcommand{\acread}{\textbf{rtv}}
\newcommand{\acread}{\textbf{qry}}
%\newcommand{\acreadp}[2]{\ensuremath{\mathbf{rtv}(#1)@#2}}
\newcommand{\acreadp}[2]{\ensuremath{\mathbf{qry}(#1)@#2}}
%\newcommand{\acmove}{\textbf{exec}}
\newcommand{\acexec}{\textbf{exec}}
\newcommand{\aceval}{\textbf{eval}}
%\newcommand{\acmovep}[2]{\ensuremath{\mathbf{exec}(#1)@#2}}
%\newcommand{\acexecp}[2]{\ensuremath{\mathbf{exec}(#1)@#2}}
\newcommand{\acexecp}[1]{\ensuremath{\mathbf{exec}(#1)}}
\newcommand{\acnewold}{\textbf{newloc}}
\newcommand{\acnew}{\textbf{new}}
\newcommand{\acnewp}[1]{\ensuremath{\mathbf{new}(#1)}}
\newcommand{\acfresh}{\textbf{fresh}}
\newcommand{\acfreshp}[1]{\ensuremath{\mathbf{fresh}(#1)}}
\newcommand{\acenforce}{\textbf{enforce}}
\newcommand{\acenforcep}[1]{\ensuremath{\mathbf{enforce}(#1)}}

\newcommand{\procnil}{\ensuremath{\textbf{nil}}}
%\newcommand{\mmif}{\textbf{if}}
%\newcommand{\mmthen}{\textbf{then}}
%\newcommand{\mmelse}{\textbf{else}}
%\newcommand{\mmfor}{\textbf{for}}
%\newcommand{\mmforall}{\textbf{forall}}
%\newcommand{\mmwhile}{\textbf{while}}
%\newcommand{\mmdo}{\textbf{do}}

\newcommand{\naddr}{n}              % network addresses
\newcommand{\naddrbis}{m}              % network addresses
\newcommand{\vaddr}{x}              % variable for network addresses
\newcommand{\nvaddr}{c}          % network addresses or variable
\newcommand{\self}{\mathsf{self}}   % local address
\newcommand{\super}{\mathsf{super}}
\newcommand{\predic}[1]{\mathsf{#1}}

\newcommand{\pair}[1]{\langle #1 \rangle}

\newcommand{\bexpr}{e}               % expressions for basic values

%\newcommand{\conf}[1]{\langle {#1} \rangle}
%\newcommand{\netconf}[2]{ #1 \vdash #2 }
\newcommand{\tuplewp}[2]{#2 \triangleright \langle #1 \rangle}
\newcommand{\tuple}[1]{\langle #1 \rangle}
%\newcommand{\pol}[2]{#2 \triangleright #1}
\newcommand{\pol}{\Pi}          % policies
\newcommand{\rules}{\rho}          % rules

%\newcommand{\parcomp}{\mbox{$ \mid $}}          % parallel among components
\newcommand{\parp}[1]{\parallel_{{}_{#1}}}          % parameterised parallel
\newcommand{\parcompp}{\parp{\pol}}                 % generic policy parameterised parallel
\newcommand{\parcomp}{\parallel}                    % parallel
\newcommand{\choice}{\: + \:}

\newcommand{\ecs}{ {\bf 0 } }                   % empty set of components
\newcommand{\knw}{\mathcal{K}}                      % knowledge representation + management
%\newcommand{\loccomp}[3]{ #1 ::_{{}_{#2}}^{{}_{#3}} }       % component
\newcommand{\loccomp}[2]{ \interf{#1}[#2] }                  % component
\newcommand{\ens}[3]{ #1 ::_{{}_{#2}}^{{}_{#3}} }  % ensemble
%\newcommand{\ens}[2]{ #1[#2] }                      % ensemble
\newcommand{\interf}[1]{\ensuremath{\mathbf{\cal #1}}}
\newcommand{\interfm}[1]{\cal #1}

\newcommand{\enspred}[1]{\mathsf{#1}}

\newcommand{\res}[1]{(\nu #1)}                  % name anonymiser
\newcommand{\reslab}[1]{\nu #1}                  % name anonymiser
\newcommand{\ren}[1]{\{#1\}}                    % name renaming
\newcommand{\penf}[1]{[#1]}                     % policy enforcement
\newcommand{\broadc}[1]{(#1 \rightrightarrows)} % broadcast
\newcommand{\auth}[2]{#1[\,#2\,]}                     % autonomic handling

\newcommand{\defi}{\stackrel{def}{=}}
\newcommand{\define}{\triangleq}
\newcommand{\wt}{\widetilde}

%\newcommand{\sep}{\ \big |\ }
\newcommand{\sep}{\ \ \left|
                      \begin{array}{l}
                      \\
                      \end{array}
                   \right.}

\newcommand{\Act}{{\cal L}}     % set of visible actions

%%% OPERATIONAL SEMANTICS

\newcommand{\valt}[2]{\mbox{${\mathcal E}[\![ \: #1 \: ]\!]_{#2}$}}
%\newcommand{\sub}[2]{\{\raisebox{.5ex}{\small$#1$}\! / \!\mbox{\small$#2$}\}} % name substitution
\newcommand{\sub}[3]{ #1 \{\raisebox{.1ex}{\small$#2$}\! / \!\mbox{\small $#3$}\}} % name substitution
\newcommand{\mumatch}{match}

\newcommand{\n}[1]{{\mathit{n}}(#1)}    % names
\newcommand{\fn}[1]{{\mathit{fn}}(#1)}    % free names
\newcommand{\bn}[1]{{\mathit{bn}}(#1)}    % bound names
\newcommand{\var}[1]{{\mathit{v}}(#1)}    % variables
\newcommand{\fv}[1]{{\mathit{fv}}(#1)}    % free variables
\newcommand{\bv}[1]{{\mathit{bv}}(#1)}    % bound variables

\newcommand{\alfaconv}{=_{\alpha}}
\newcommand{\reduce}{\longmapsto}

\newcommand{\thenstr}{\prlaw{Then}}
\newcommand{\elsestr}{\prlaw{Else}}
\newcommand{\alphastr}{\prlaw{Alpha}}
\newcommand{\callstr}{\prlaw{Rec}}
\newcommand{\defstr}{\prlaw{PDef}}
\newcommand{\ridstr}{\prlaw{RNil}}
\newcommand{\rnodestr}{\prlaw{RNode}}
\newcommand{\rnodepstr}{\prlaw{RNode'}}
\newcommand{\rcomstr}{\prlaw{RCom}}
\newcommand{\pidstr}{\prlaw{PZero}}
\newcommand{\pcomstr}{\prlaw{PCom}}
\newcommand{\passstr}{\prlaw{PAss}}
\newcommand{\absstr}{\prlaw{Abs}}
\newcommand{\clonestr}{\prlaw{Clone}}
\newcommand{\extstr}{\prlaw{Ext}}

\newcommand{\prlaw}[1]{(\mbox{{\sc{#1}}})}  % to give names to rules


%%%%%%%%%%%%% CC-PI %%%%%%%%%%%%%%%%%%%%%%%%%%%%%%%%%
    \newcommand{\per}{\times} % operazione prodotto di un c-semiring
    \newcommand{\zero}{0} % elemento 0 del c-semiring
    \newcommand{\uno}{1} % elemento 1 del c-semiring
    \newcommand{\isCons}[1]{isCons(#1)}     % consistency predicate
    \newcommand{\entail}[2]{#1 \vdash #2}   % entailment predicate
\newcommand{\getSol}[1]{getSol(#1)}
\newcommand{\unitEl}{0} % elemento unita' del c-semiring
\newcommand{\absEl}{1} % elemento assorbente  del c-semiring
\newcommand{\noBlockTell}[3]{\ccaction{tell}\ #1\,\{#2\}\{#3\}}
    \newcommand{\ccunion}{\uplus}
\newcommand{\store}[1]{\sout{\single{c}}{\arr{#1}}} % store dei constraints
\newcommand{\ps}{\partner_s} % partner dello store
\newcommand{\opreq}{\op_{get}} % operation di request dello store
\newcommand{\opres}{\op_{set}} % operation di response dello store
    \newcommand{\ccaction}[1]{\mathtt{#1}} % font per le azioni cc-pi
\newcommand{\ccenc}[1]{\encod{#1}} % encoding azioni
\newcommand{\ccencServ}[1]{\encod{#1}} % encoding servizi
\newcommand{\ccvar}[1]{\mathtt{#1}} %constrained variables
    \newcommand{\cvar}[1]{cv(#1)} % constrained variables di un termine
\newcommand{\ccmark}[1]{mark(#1)} % marca un vincolo con una tilde
\newcommand{\ccmarked}[1]{#1^{\vdash}} % vincolo marchiato
\newcommand{\ccunmark}[1]{unmark(#1)} % elimina la tilde
\newcommand{\client}{\textsf{C}}
\newcommand{\registry}{\textsf{R}}
\newcommand{\provider}[1]{\textsf{P}_{#1}}
\newcommand{\thirdparty}[1]{\textsf{T}_{#1}}

%%% Arrows %%%
\makeatletter
\def \rightarrowfill{\m@th\mathord{\smash-}\mkern-6mu%
  \cleaders\hbox{$\mkern-2mu\mathord{\smash-}\mkern-2mu$}\hfill
  \mkern-6mu\mathord\rightarrow}
\makeatother
\newcommand{\transition}[1]{\overstackrel{\rightarrowfill}{\ \ #1\ \ \phantom{\dag}}} % labelled transition
\newcommand{\redarrow}[1]{\succ\!\overstackrel{\rightarrowfill}{\ \ #1\ \ }} % reduction arrow

\makeatletter
\def \wrightarrowfill{\m@th\mathord{\smash=}\mkern-6mu%
  \cleaders\hbox{$\mkern-2mu\mathord{\smash=}\mkern-2mu$}\hfill
  \mkern-6mu\mathord\Rightarrow}
\makeatother
\newcommand{\weakTransition}[1]{\overstackrel{\wrightarrowfill}{\ \ #1\ \ \phantom{\dag}}} % weak labelled transition
\newcommand{\weakRed}{\Longrightarrow} % weak reduction

\makeatletter
\def \nrightarrowfill{\m@th\mathord{\smash-}\mkern-6mu%
  \cleaders\hbox{$\mkern-2mu\mathord{\smash-}\mkern-2mu$}\hfill
  \mkern-6mu\mathord\nrightarrow}
\makeatother
\newcommand{\ntransitionShort}[1]{\overstackrel{\rightarrowfill}{\ \ #1\ \ \phantom{\dag}}
            \!\!\!\!\!\!\!\!\! /} % negated labelled transition
\newcommand{\ntransitionLong}[1]{\overstackrel{\rightarrowfill}{\ \ #1\ \ \phantom{\dag}}
            \!\!\!\!\!\!\!\!\!\!\!\!\!\!\! /} % negated labelled transition


\newcommand{\plabel}[2]{\left\langle~#1~,~#2~\right\rangle}
\newcommand{\authl}[2]{#1[\,#2\,]}
\newcommand{\punit}{\circ}
\newcommand{\procarrow}[1]{\rMapsto{#1}}
\newcommand{\doPutR}[3]{(#3)#1\triangleright #2}
\newcommand{\docomm}[2]{#1 \diamond #2}
\newcommand{\doPut}[2]{#1\triangleright #2}
\newcommand{\doPutCS}[3]{#1\triangleright_{#3} #2}
\newcommand{\doPutTauR}[2]{(#2)#1\triangleright \tau}
\newcommand{\doPutTau}[1]{#1\triangleright \tau}
\newcommand{\doGet}[2]{#1\triangleleft #2}
\newcommand{\doRead}[2]{#1\blacktriangleleft #2}
\newcommand{\doExec}[2]{#1\blacktriangleright #2}
\newcommand{\acceptPut}[2]{#1\,\bar{\triangleright}\, #2}
\newcommand{\acceptGet}[2]{#1\,\bar{\triangleleft}\, #2}
\newcommand{\acceptRead}[2]{#1\,\bar{\blacktriangleleft}\, #2}
\newcommand{\acceptExec}[2]{#1\otimes #2}
\newcommand{\bit}[1]{[ #1 \rangle\!}
\newcommand{\attrens}{\mathsf{ens}}
\newcommand{\attrblg}{\mathsf{belong}}
\newcommand{\ensarrow}[1]{\rTo{#1}}

\def \overunderstackrel#1#2#3{\mathrel{\mathop{#1}\limits^{#2}_{#3}}}
\def \overstackrel#1#2{\mathrel{\mathop{#1}\limits^{#2}}}
\def \understackrel#1#2{\mathrel{\mathop{#1}\limits_{#2}}}
\def\dlarrow#1#2{\overunderstackrel{\rightarrowfill}{#1}{#2}}
\newcommand{\ensarrowcs}[2]{\dlarrow{#1}{#2}}
%\newcommand{\partarrow}[2]{\overunderstackrel{\rightarrowfill}{\ \ \mbox{\scriptsize $#1$}\ \ }{\ \mbox{\scriptsize $#2$}\ }\!\!\!>}
%\newcommand{\ensarrowcs}[2]{\partarrow{#1}{#2}}

%\newcommand{\redarr}{\rightarrowtail}
\newcommand{\redarr}{\succ\!\overstackrel{\rightarrowfill}{\ \ \ \ }} % reduction arrow
\newcommand{\polren}{\succ}
\newcommand{\polrenm}{\succ_{\mathsf{M}}}
\newcommand{\allow}{\vdash}


%%%%%%%%%%  LOGIC

\newcommand{\ttact}{\textit{tt}}
\newcommand{\ffact}{\textit{f\mbox{}f}}
\newcommand{\ltrue}{\textit{true\/}}
\newcommand{\lfalse}{\textit{false\/}}
\newcommand{\dmnd}[1]{\mathrm{<}#1\mathrm{>}}
\newcommand{\bx}[1]{\mathrm{[}#1\mathrm{]}}



%%%%%%%%%% CONTROLLED ALGEBRA
\newcommand{\ipred}{\pol_{\oplus}}
\newcommand{\cpred}[2]{\pol_{\otimes}[#1,#2]}
\newcommand{\ctpred}[3]{\pol_{#2}[#1,#3]}
\newcommand{\spred}[2]{\pol_{S}[#1,#2]}
\newcommand{\valn}[2]{\mbox{${\mathcal N}[\![ \: #1 \: ]\!]_{#2}$}}
\newcommand{\valp}[2]{\mbox{${\mathcal P}[\![ \: #1 \: ]\!]_{#2}$}}
\newcommand{\calt}{\mathcal{T}}
